\section{Introduction}

%One of the values of the Hypertext Transfer Protocol (HTTP) often known as the World
%Wide Web is that one document is linked to other documents and relevant information
%can be obtained through the links. Similarly, the semantic web is a relatively new
%field of research and a fledgling set of standards that attempts to create a web of
%linked data in standardized format to help machine readability and processing. The
%web and the semantic web are tightly woven due to that much of semantic data are
%existing data on the web migrated to the new format or obtained by annotating
%the existing web with semantic information .
%
%Much of the semantic web data describe geospatial information, such as location of
%places, and geospatial information system that access this information is able to
%perform
%
%---

%Semantic web is a fledgling field of research which aims to create a web of machine readable and processable data. Resource Description Framework (RDF) is a standard for describing related entities in the semantic web, i.e., Friend-of-A-Friend\footnote{http://www.foaf-project.prg/}.

%Each property of an entity is described as a triple (S, P, O). A triple consists of a subject, denoting the entity, a predicate, meaning a property of the entity, and an object, the value of that property. The object of a triple can be either a literal or the subject of another triple, therefore creating a link between the two entities. Such a system that manages a set of triple is called a \emph{triple store}.

With the significant development of the GPS navigation technologies and the rapidly growing popularity of location-based services, today various spatial/geographic data abounds on the Semantic web, such as OpenStreetMap\footnote{http://www.openstreetmap.org/} and LinkedGeoData\footnote{http://linkedgeodata.org/}. As a result, how to efficiently manage spatial data on triple stores has attracted more and more interests from researcher. However, there are currently limited solutions providing efficient management of spatial data in this field. In this demonstration, we present Geo-Store~\cite{journals/internet/KuCWL}, a novel spatially-aware data management system built on top of RDF-3X~\cite{DBLP:journals/vldb/NeumannW10}, a state-of-art scalable RDF data management system. By the augmentation of the standard SPARQL language with spatial query filters, Geo-Store is able to analyze and process complex queries with common spatial constraints. Those spatial filters allow user to formulate spatial relationships or constraints into standard SPARQL queries. Afterwards, spatial query filters are analyzed and evaluated based on our proposed hashing scheme, Spatially-Aware Hashing (SAH). In SAH, spatial data are pre-processed and encoded with their Hilbert values by using space-filling curves to accelerate the processing of spatial queries without any compromise in accuracy. In addition to utilizing the SAH scheme to speed up the spatial query processing, Geo-Store provides a graphical query builder interface for users to easily compose spatial and non-spatial constraints to form SPARQL queries.  